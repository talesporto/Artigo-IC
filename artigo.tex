\documentclass[bacharelado]{unb-cic}
\usepackage[american,brazil]{babel}
\usepackage[T1]{fontenc}
\usepackage{indentfirst}
\usepackage{natbib}
\usepackage{xcolor,graphicx,url}
\usepackage[utf8]{inputenc}


\bibpunct[; ]{(}{)}{,}{a}{}{;}%muda colchetes para parênteses

% definições prévias do documento
\title{Simpsons segundo os apocalípticos}

\diamesano{2}{maio}{2011}

\autor

\begin{document}

\maketitle

\textual

\chapter{Introdução}

Criada por Matt Groening, a famosa série de tv americana ``Os Simpsons'', exibida desde 1989 nos Estados Unidos e desde 1991 no Brasil, mostra a vida do típico cidadão americano respresentada pelo seu protagonista, Homer Simpson.
Com grande aceitação popular a série chegou a sua 22ª temporada com seus 486 episódios levando o seu protagonista a ser eleito ``O maior americano de todos os tempos'' pelos próprios americanos no ano de 2003 através de uma pesquisa realizada pelo site da BBC, ficando, inclusive, na frente de personalidades como Abraham Lincoln e Matin Luther King Jr \cite{pasquolotto}.

A série é composta por mais quatro personagens principais sendos estes, Marge, Bart, Lisa e Maggie Simpson. O artigo apresenta uma analíse apartir do episódio 112, ``O tarado Homer''(\textit{Homer Badman}). Este episódio tematiza as discussoẽs sobre a formação da opinião pública a partir da televisão. Exemplificando a formação deturpada da opinião a partir da exibição de fatos incompletos e reorganizados \cite{pasquolotto}. A partir da desconstrução da narrativa é possível notar mensagens ímplicitas que são apresentadas de forma lúdica e bem humorada levando a formação de opinião.

As análises serão feitas em cima do conceito de apocalíptico, apresentado por Umberto Eco, onde é possível ver a manipulação da informação modificando a cultura de massa \cite{eco}.

\chapter{Apocalíptico}

\chapter{Apocalípticos X Simpsons}

\chapter{Conclusão}

\postextual

\bibliographystyle{plain}

\bibliography{bibliografia}

\end{document}
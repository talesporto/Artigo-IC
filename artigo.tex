\documentclass[bacharelado]{unb-cic}
\usepackage[american,brazil]{babel}
\usepackage[T1]{fontenc}
\usepackage{indentfirst}
\usepackage{natbib}
\usepackage{xcolor,graphicx,url}
\usepackage[utf8]{inputenc}


\bibpunct[; ]{(}{)}{,}{a}{}{;}%muda colchetes para parênteses

% definições prévias do documento
\title{Simpsons segundo os apocalípticos}

\diamesano{2}{maio}{2011}

\autor

\begin{document}

\maketitle

\textual

\chapter{Introdução}

Os Simpsons é uma série de animação que permanece no ar desde 1989 nos Estados Unidos e desde 1991 no Brasil, sempre fazendo sátiras da vida americana e da sociedade contemporânea em geral, causando polêmicas e virando tema de livros e discussões intelectuais. Mesmo sendo um produto da mídia de massa a série tornou-se instrumento base para disciplinas acadêmicas de várias universidades Americanas como é o caso da \textit{Tufts University} que oferece o curso ``Os Simpsons e a Sociedade'' e da \textit{Siena Heightsde} que oferece o curso de filosofia e religião baseado no livro ``Os Simpsons e a Filosofia'' e serve como estímulo para fazer os alunos aprenderem. Com base nisso, percebe-se que Os Simpsons pode ser um bom objeto de estudo do campo comunicacional \cite{pasquolotto}.

Ao análisar o episódio ``O tarado Homer'', para verificar como este episódio tematiza discussões sobre a formação da opinião pública a partir da televisão, mostrando ângulos de abordagem propostos pelo episódio englobando a mídia e as funções que estas mídias exercem na contemporaneidade. Abordando assim, as lógicas de produção de sentido no que se referem à produção/criação de fatos midiáticos para serem discutidos/julgados no campo social \cite{pasquolotto}.

O referencial teórico discute sobre o espaço público e a opinião pública pela perspectiva sociológica e a formação da opinião pública a partir da mídia sob os olhares críticos de Marcondes Filho e Ramonet \cite{pasquolotto}.

As análises foram desenvolvidas a partir de uma descrição densa do episódio, pois parte-se do pressuposto de que as mensagens implícitas somente são detectadas a partir da desconstrução da narrativa. As categorias de análise são o figurino, as expressões gestuais, as cores, as trilhas sonoras e o conteúdo textual. Desta forma, através de Os Simpsons demonstra-secomo um desenho (visto de forma ingênua) pode conter mensagens implícitas de forma lúdica e bem humorada sobre a formação da opinião pública através das mídias \cite{pasquolotto}.

\subsection{Simpsons}

\chapter{Apocalípticos}
	% https://sites.google.com/site/richardromancini/umbertoecoeos"apocalípticoeintegrados"

	\begin{itemize}
		\item dirige-se a um publico heterogenio
		\item Difunde globalmente uma cultura do tipo homegênio, destruindo as características culturais próprias de cada grupo étnico~\cite{eco}; %escosses tarado
		\item O \texit{mass media} dirige-se a um público incônscio de si mesmo como grupo social caracterizado; o público, portanto, não pode manifestar exigências face a cultura de massa, mais deve sofrer-lhe as propostas sem saber que as sofre~\cite{eco};
		\item O \texit{mass media} tende a provocar emoções intensas e não mediatas; em outros termos, ao invés de simbolizar uma emoção, de representá-la, provocam-na; ao invés de a sugerirem, entregam-na já confeccionada. Típico, nesse sentido, o papel da imagem em relação ao conceito~\cite{eco};
		\item O \texit{mass media} colocado dentro de um circuito comercial, esta sujeito a ``lei da oferta e da procura''. Dão ao público, portanto, somente o que ele quer, ou, o que é pior, seguindo as leis de uma economia baseada e sustendada pela ação persuasiva da publicidade, sugerem ao público o que este deve desejar~\cite{eco};
		



		\item Evitam originalidade e adotam padrão diluidor e homogeneizante, reduzindo ao mínimo a individualidade de produtores e consumidores e das experiências~\cite{richard};
		\item Inseridos no mercado, os produtos somente dão ao público o que ele quer, mas a publicidade sugere às pessoas o que desejar;
		\item Estimulam visão passiva e acrítica do mundo, ao igualar todas as coisas e propostas~\cite{richard};;
		\item São conservadores por trabalhar com opiniões comuns, favorecendo o conformismo no campo dos costumes, valores etc~\cite{richard};.
		\item Esvaziam o conteúdo da “cultura superior” ao veiculá-la e tomam lugar de uma cultura genuinamente popular, mas os produtos dos \texit{media mass} – impostos de cima – não têm qualidades da “cultura popular” (humor, vitalidade etc.)~\cite{richard};;
		\item São a “superestrutura do regime capitalista”, usada com fins de controle e manipulação das consciências~\cite{richard};.
	\end{itemize}

	
% https://sites.google.com/site/richardromancini/umbertoecoeos%22apocal%C3%ADpticoeintegrados%22

\chapter{Apocalípticos X Simpsons}


\chapter{Conclusão}
teste 

\postextual

\bibliographystyle{plain}

\bibliography{bibliografia}

\end{document}
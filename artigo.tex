\documentclass[bacharelado]{unb-cic}
\usepackage[american,brazil]{babel}
\usepackage[T1]{fontenc}
\usepackage{indentfirst}
\usepackage{natbib}
\usepackage{xcolor,graphicx,url}
\usepackage[utf8]{inputenc}


\bibpunct[; ]{(}{)}{,}{a}{}{;}%muda colchetes para parênteses

% definições prévias do documento
\title{A midia de massa segundo os apocalípticos analisando o episódio ``O tarado Homer''}

\diamesano{2}{maio}{2011}

\autor

\begin{document}

\maketitle

\textual

\chapter{A série}

Criada em 1989 por Matt Growning, retrata a vida dos Simpsons, uma família norte americana tipicamente normal, composta por:

\section{Hommer}

Pai de família e inspetor de segurança na usina de energia nuclear de Springfield. Criado por seu pai, Abe, que tentava compensar a ausência da mãe de Homer, uma hippie radical, Homer se formou em último lugar no secundário e deu um jeito de ganhar distinção sendo o mais antigo funcionário da usina. Junto com sua namorada do secundário, a Marge Bouvier, Homer se estabeleceu em Evergreen Terrace, um bairro de classe média baixa de Springfield, onde cria seus três filhos. Homer gosta de cerveja, doces, roscas, as costeletas de porco de Marge, de assistir o Bee Guy no canal espanhol e de jogar boliche. Suas antipatias incluem seu patrão, o Sr. Burns e seu vizinho, Ned Flanders \cite{simpsonsBrasil}.

\section{Marge}

Dona de casa cujo maior orgulho são seus três filhos e o seu marido Homer, o qual ela sempre apóia, mesmo que ele faça suas ``burradas''. Marge também tem um relacionamento forte com suas irmãs, Patty e Selma, e com seu sogro, Abe Simpson. Além de suas obrigações em casa, Marge já teve interesse por uma série de carreiras, desde oficial de polícia até vendedora de Pretzels \cite{simpsonsBrasil}.

\section{Bart}

Filho mais velho, está no 4º ano e adora passar trotes telefônicos para a Taverna do Moe.
Também gosta de brincar com seu cachorro e com Milhouse. Podemos dizer que Bart é um garoto de sorte, pois já realizou uma série de sonhos: estrelou sua própria série de TV (com seu ídolo, Krusty, o Palhaço), localizou e deu nome a um cometa mortal que quase destruiu sua cidade, e quase atrapalhou o papel do Caídaço Boy no filme do Homem Radioativo. Mas sua maior travessura foi quando fingiu ser um garoto que estava no fundo do poço, fazendo com que a população ficasse em estado de choque \cite{simpsonsBrasil}.

\section{Lisa}

Garota muito inteligente que espera ansiosa pela faculdade. Apesar de ter apenas oito anos, ela lê perfeitamente bem e já escreveu uma série de ensaios de qualidade excepcional, com um dos quais conseguiu uma viagem gratuita para sua família para Washington. Suas atividades preferidas são tocar seu saxofone, ir à escola e ler revistas para Garotos. Fã de Malibu Stacy, Lisa já tentou sem sucesso criar sua própria boneca falante, a “Lisa Coração-de-Leão”. Além disso, é vegetariana e sempre quis ter um pônei \cite{simpsonsBrasil}.

\section{Maggie}

Com apenas 1 ano de idade já aprendeu a soletrar o próprio nome, já andou pela cidade de Springfield sozinha e atirou no homem mais rico de Springfield, o Sr. Burns. Ela não larga a sua chupeta e tem raiva de Gerald (o bebê de uma só sobrancelha) \cite{simpsonsBrasil}.


A série obteve grande aceitação popular e atualmente encontra-se na 22a temporada, tendo seu principal personagem, Hommer Simpson, que foi eleito “o maior americano de todos os tempos” pelo próprio povo americano em uma pesquisa realizada em 2003 pela BBC.
Os roteiristas fazem uso de diversas formas de linguagem para passar suas mensagens e contarem suas histórias. O ponto mais evidente na série é a sátira com temas como: política, religião, artes, esportes, álcool, legislação e atualidades, feito com total exagero para criar o efeito cômico que uma série de humor deve ter.
É um produto da mídia que cativa populares e intelectuais e não tem vergonha de criticar os outros e a si mesmos. Serve principalmente como espelho de uma sociedade preguiçosa, e acostumada apenas a receber a informação pronta e apurada, como uma verdade única. O foco principal é não aceitar qualquer coisa como verdade apenas porque uma figura de autoridade assim declarou. A série é engraçada, mas o uso que ela faz da sátira vai muito além do humor. Ela busca a verdade numa cidade cheia de corrupção e mentira.
Atualmente, é exibida para vários países e, devido às sátiras constantemente realizadas, tornou-se alvo de várias críticas e até mesmo censuras. Frequentemente aproveita o sucesso para lançar vários artefatos com a marca da série, incluindo livros, revistas, bonecos, jogos, roupas, acessórios, DVD’s e até mesmo um filme, podendo, portanto, ser considerada um produto da mídia de massa.

\chapter{O episódio}

Foi escolhido o episódio “O tarado Homer” (3), cuja temática refere-se à formação da opinião pública por meio da televisão, exemplificando a formação deturpada da opinião a partir da exibição de fatos incompletos e reorganizados (2).

“Homer resolve ir para uma feira anual de doces na sua cidade e contrata uma babá para ficar com as crianças, já que sua esposa precisa ir junto para ajudar a carregar os doces. Na feira, Homer encontra uma goma de mascar muito rara. Fanático por doces, Homer furta a goma e sai correndo. De volta em casa, e já tarde da noite, Homer percebe que não encontra a goma preciosa, conhecida também como Vênus. Mas sua esposa o obriga a levar a babá para casa. Quando a babá sai do carro, Homer percebe que a goma preciosa ficou grudada na calça da moça. Ingenuamente, Homer pega a goma, e assusta a babá, que sai correndo. Sem se importar e entender nada, Homer agradece. Na manhã seguinte, a casa da família Simpson amanhece repleta de manifestantes. Homer pensa que é por ter furtado a goma preciosa, mas logo percebe o que estava acontecendo, ele vê a babá gritando no meio da multidão: ‘Alí está ele! Ali está o homem que abusou sexualmente de mim!’. Espantado com a acusação, Homer tenta se defender, mas a população já estava com a sua opinião formada a respeito dele.
Na tentativa de influenciar a opinião pública a seu favor, aceita o convite de um repórter e decide aparecer na TV para dar sua versão da história. No estúdio de gravação, Homer contou: ‘Eu tive que levar a babá pra casa, aí eu vi que ela estava sentada em cima da goma Vênus, então eu agarrei o doce no traseiro dela e... Oh, só de pensar naquele doce, doce docinho. Eu só queria ter outra agora mesmo! Mas o mais importante é...’ o repórter interrompe: ‘Foi muito bom, Sr. Simpson. Já temos tudo o que precisávamos’.
A entrevista foi gravada e editada pelos produtores do programa para fazer Homer parecer um homem pervertido e sem caráter, em troca de muita audiência com um caso polêmico. O anúncio da entrevista dizia: ‘Ela era uma universitária honrada que dedicava a vida às crianças, até à noite em que um canalha pervertido chamado Homer Simpson, lhe deu uma carona para a depravação’ e entrevista editada e explicitamente tendenciosa mostrava um Homer Simpson muito diferente daquele da entrevista verdadeira: ‘Eu tive que levar a babá em casa, e notei que ela estava sentada no... doce traseiro... por isso eu agarrei o.. doce traseiro... só em pensar no doce traseiro eu... quem me dera poder comer o... doce traseiro’ repórter: ‘Então, Sr. Simpson, admite ter agarrado o traseiro dela. O que tem a dizer em sua defesa? Sr. Simpson, o seu silêncio só o vai incriminar mais. Não, Sr. Simpson. Não vire a sua raiva contra mim. Afaste-se! Afaste-se! Sr. Simpson, não!’ após a entrevista o programa exibiu a mensagem: ‘Não ouve montagem’.
O programa influenciou os telespectadores a ficarem contra Homer, e ter uma mesma idéia sobre os fatos, a de que ele era um maníaco sexual. Todos os programas sensacionalistas cobriam o caso, fazendo do que seria uma causa judicial, um espetáculo midiático.
Ninguém se lembrou que hoje em dia a informação televisada é essencialmente um divertimento, um espetáculo. Que ela se nutre fundamentalmente de sangue, de violência e de mortes. E isto mais ainda devido à concorrência desenfreada entre as emissoras que obrigam os jornalistas a buscar o sensacional a qualquer preço, a querer ser, cada um deles, o primeiro no local e a enviar de lá imagens fortes. Esses imperativos não levam em conta o fato de que às vezes é materialmente impossível verificar se não se é vitima de uma intoxicação, de uma manipulação, e que os repórteres não dispõem de tempo para analisar seriamente a situação. (RAMONET, 1999, p.101 à 102)
A força da mídia, e principalmente da televisão é tão forte que o próprio acusado, Homer Simpson, acredita, por um momento, que realmente é culpado: ‘talvez a televisão esteja certa, ela sempre está certa’. Seus filhos, apesar de acreditarem na honestidade do pai também se mostram fracos perante a televisão: ‘é difícil não acreditar na televisão, ela nos educou muito mais do que você’.
Para tentar dar a versão correta dos fatos, os Simpsons decidem usar a televisão pública local da cidade, mas como é um canal fraco, sem poderes, sem dinheiro e sem programação sensacionalista, ninguém assistiu a versão de Homer, mostrando que a junção de comunicação e poder, pode ser um veículo perigoso.
No final do episódio, Homer é salvo pelo zelador Willie que tinha gravado coincidentemente a cena em que Homer supostamente assediava a babá. A televisão e, principalmente o programa que usou a imagem de Homer para condená-lo, fez uma rápida retratação. Dias mais tarde, Homer assistia ao programa que o avia condenado injustamente. Nele o zelador Willie era acusado de ser um criminoso. Homer fica revoltado: ‘Escute o que eles dizem, esse cara é mau!’ a mulher de Homer fica indignada: ‘Essa experiência não ensinou você a não acreditar em tudo que ouve?’ e então Homer, ironicamente, responde: ‘Marge, querida, eu não aprendi nada!’. E então beija a TV.

\chapter{Análise do episódio}

As análises serão feitas a partir de um ponto de vista apocalíptico, conforme os conceitos da mídia de massa apresentados por Umberto Eco, com os quais é possível criticar a manipulação da informação modificando a cultura de massa (1).



Apresentando de forma ``rasa'' a notícia, as mídias de massa capturam o nível superficial da nossa atenção. No episódio é mostrado um fato que ocorre com uma certa frequência nas mídias de massa. Se trata da manipulação da informação para que a mesma condiza com as opiniões do gerador da informação, e também com os seus interesses. Também é mostrado no episódio, e fica claro que os receptores da informação acreditaram na primeira história que ouviram e não buscaram se aprofundar e obter novas fontes para averiguar a veracidade da história. O que gerou um trantorno enorme para a vítima da calunia que teve que provar sua inocência, mesmo as acusações sendo infundadas e inconfiáveis.



A mídia de massa é colocada dentro de um circuito comercial e está sujeita à “lei da oferta e da procura”, sendo que se produz informação para atender o público que está avido por consumi-la, como é o caso de muitas revistas de fofocas e como foi exemplificado no episódio.

\chapter{Conclusão}

\postextual

\bibliographystyle{plain}

\bibliography{bibliografia}

\end{document}
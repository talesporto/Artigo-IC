\documentclass[bacharelado]{unb-cic}
\usepackage[american,brazil]{babel}
\usepackage[T1]{fontenc}
\usepackage{indentfirst}
\usepackage{natbib}
\usepackage{xcolor,graphicx,url}
\usepackage[utf8]{inputenc}


\bibpunct[; ]{(}{)}{,}{a}{}{;}%muda colchetes para parênteses

% definições prévias do documento
\title{Simpsons segundo os Apocalípticos}

\diamesano{2}{maio}{2011}

\autor

\begin{document}

\maketitle

\textual

\chapter{Introdução}

Criada por Matt Groening, a famosa série de tv americana ``Os Simpsons'', exibida desde 1989 nos Estados Unidos e desde 1991 no Brasil, mostra a vida do típico cidadão americano respresentada pelo seu protagonista, Homer Simpson \cite{personagensSimpsons}.
Com grande aceitação popular a série chegou a sua 22ª temporada com seus 486 episódios levando o seu protagonista a ser eleito ``O maior americano de todos os tempos'' pelos próprios americanos no ano de 2003 através de uma pesquisa realizada pelo site da BBC, ficando, inclusive, na frente de personalidades como Abraham Lincoln e Matin Luther King Jr \cite{pasquolotto}.

 \begin{figure}[h]
    	\centering \includegraphics[scale=0.3]{figuras/simpsons1.jpg}
    	\caption{Simpsons imitando os beatles.}
	\label{beatles} 
  \end{figure}

A série é um produto da mídia de massa que faz sátiras da vida americana e da sociedade conteporânea em geral. Por causa dessas sátiras, está no meio de várias polêmicas e já virou tema de livros e discussões intelectuais. 

Uma dessas polêmicas envolve o Brasil. No episódio \texit{FEitiço de Lisa}...

Há livros que discutem temas como psicologia, filosofia, política, religião, relações sociais e o mundo contemporâneo:

  \begin{itemize}
      \item \textit{The World According to the Simpsons - What Our Favorite TV Family Says About Life, Love, And the Pursuit of the Perfect Donut} de Steven Keslowitz;
      \item \textit{Machiavelli Meets Mayor Quimby - Political Commentary in the First Season of The Simpsons} de Natham Thoms;
      \item \textit{Planet Simpson - How a Cartoon Masterpiece Defined a Generation} de Chris Turner;
      \item \textit{De Olho em Springfield} de Johan L. Lagger;
  \end{itemize}

A série é composta por mais quatro personagens principais sendos estes, Marge, Bart, Lisa e Maggie Simpson, a maioria mostrados na figura \ref{beatles}. O artigo apresenta uma analíse apartir do episódio 112, ``O tarado Homer''(\textit{Homer Badman}) \cite{episodiosSimpsons}. Este episódio tematiza as discussoẽs sobre a formação da opinião pública a partir da televisão. Exemplificando a formação deturpada da opinião a partir da exibição de fatos incompletos e reorganizados \cite{pasquolotto}. A partir da desconstrução da narrativa é possível notar mensagens ímplicitas que são apresentadas de forma lúdica e bem humorada levando a formação de opinião.

\chapter{Os Simpsons segundo os Apocalípticos}

As análises serão feitas em cima dos conceitos da mídia de massa segundos os Apocalípticos, apresentado por Umberto Eco, onde é possível ver a manipulação da informação modificando a cultura de massa \cite{eco}.

A mídia de massa se dirige a um público heterogênio, difundindo uma cultura do tipo homogênio, destruindo as características culturais de cada grupo étnico. Com isso, impõe símbolos e mitos de fácil universalidade \cite{eco}. Esses símbolos são facilmente reconhecíveis através das sátiras que ridicularizam os problemas sociais existentes em outros países criando imagen sócio-culturais exageradas. Isso pode ser exemplificado nos seguintes episódios":

	\begin{itemize}
      \item \textit{O Tarado Homer}: em uma parte do episódio, cria um estereótipo do Escocês solteiro, onde diz que todos possuem o hob de filmar casais no carro em fitas de video cassete;
      \item \textit{Feitiço de Lisa}: define o Rio de Janeiro como uma cidade violenta, onde só tem favelas e animais descontrolados no meio da rua;
      \item \textit{De Muitos, Wiggum}:;
  \end{itemize}


	

% 	\begin{itemize}
% 		\item dirige-se a um publico heterogenio
% 		\item Difunde globalmente uma cultura do tipo homegênio, destruindo as características culturais próprias de cada grupo étnico~\cite{eco}; %escosses tarado
% 		\item O \texit{mass media} dirige-se a um público incônscio de si mesmo como grupo social caracterizado; o público, portanto, não pode manifestar exigências face a cultura de massa, mais deve sofrer-lhe as propostas sem saber que as sofre~\cite{eco};
% 		\item O \texit{mass media} tende a provocar emoções intensas e não mediatas; em outros termos, ao invés de simbolizar uma emoção, de representá-la, provocam-na; ao invés de a sugerirem, entregam-na já confeccionada. Típico, nesse sentido, o papel da imagem em relação ao conceito~\cite{eco};
% 		\item O \texit{mass media} colocado dentro de um circuito comercial, esta sujeito a ``lei da oferta e da procura''. Dão ao público, portanto, somente o que ele quer, ou, o que é pior, seguindo as leis de uma economia baseada e sustendada pela ação persuasiva da publicidade, sugerem ao público o que este deve desejar~\cite{eco};
% 		 \item O \texit{mass media} encorajam uma visão passiva e acrítica do mundo. Desencorajam-se o esforço pessoal pela posse de uma nova experiência~\cite{eco}.
% 		 \item Feitos para o entreterimento e o lazer, são estudados para empenharem unicamente o nível superficial da nossa atenção. De saída, viciam a nossa atitude, e por isso, mesmo uma sinfonia, ouvir através de um disco ou do rádio, será fruída do modo mais epidérmico, com indicação de um motivo assobiável, e não como um organismo estético a ser penetrado em profundidade, mediante a uma atenção exclusiva e fiél~\cite{eco}.
% 		 \item O \texit{mass media} tende a impor símbolos e mitos de fácil universalidade, criando ``tipos'' prontamente reconhecíveis e por isso reduzem ao mínimo a inidividualidade e o caráter concreto não só de nossas experiências como de nossas imagens, através das quais deveriamos realizar experiências~\cite{eco};
% % https://sites.google.com/site/richardromancini/umbertoecoeos"apocalípticoeintegrados"
% 		\item Evitam originalidade e adotam padrão diluidor e homogeneizante, reduzindo ao mínimo a individualidade de produtores e consumidores e das experiências~\cite{richard};
% 		\item Inseridos no mercado, os produtos somente dão ao público o que ele quer, mas a publicidade sugere às pessoas o que desejar;
% 		\item Estimulam visão passiva e acrítica do mundo, ao igualar todas as coisas e propostas~\cite{richard};
% 		\item São a “superestrutura do regime capitalista”, usada com fins de controle e manipulação das consciências~\cite{richard};.
% 	\end{itemize}

	
% % https://sites.google.com/site/richardromancini/umbertoecoeos%22apocal%C3%ADpticoeintegrados%22



\chapter{Conclusão}

\postextual

\bibliographystyle{plain}

\bibliography{bibliografia}

\end{document}
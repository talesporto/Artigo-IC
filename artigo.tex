\documentclass[bacharelado]{unb-cic}
\usepackage[american,brazil]{babel}
\usepackage[T1]{fontenc}
\usepackage{indentfirst}
\usepackage{natbib}
\usepackage{xcolor,graphicx,url}
\usepackage[utf8]{inputenc}


\bibpunct[; ]{(}{)}{,}{a}{}{;}%muda colchetes para parênteses

% definições prévias do documento
\title{Simpsons segundo os apocalípticos}

\diamesano{2}{maio}{2011}

\autor

\begin{document}

\maketitle

\textual

\chapter{Introdução}

Os Simpsons é uma série de animação que permanece no ar desde 1989 nos Estados Unidos e desde 1991 no Brasil, sempre fazendo sátiras da vida americana e da sociedade contemporânea em geral, causando polêmicas e virando tema de livros e discussões intelectuais. Mesmo sendo um produto da mídia de massa a série tornou-se instrumento base para disciplinas acadêmicas de várias universidades Americanas como é o caso da \textit{Tufts University} que oferece o curso ``Os Simpsons e a Sociedade'' e da \textit{Siena Heightsde} que oferece o curso de filosofia e religião baseado no livro ``Os Simpsons e a Filosofia'' e serve como estímulo para fazer os alunos aprenderem. Com base nisso, percebe-se que Os Simpsons pode ser um bom objeto de estudo do campo comunicacional \cite{pasquolotto}.

Ao análisar o episódio ``O tarado Homer'', para verificar como este episódio tematiza discussões sobre a formação da opinião pública a partir da televisão, mostrando ângulos de abordagem propostos pelo episódio englobando a mídia e as funções que estas mídias exercem na contemporaneidade. Abordando assim, as lógicas de produção de sentido no que se referem à produção/criação de fatos midiáticos para serem discutidos/julgados no campo social \cite{pasquolotto}.

O referencial teórico discute sobre o espaço público e a opinião pública pela perspectiva sociológica e a formação da opinião pública a partir da mídia sob os olhares críticos de Marcondes Filho e Ramonet \cite{pasquolotto}.

As análises foram desenvolvidas a partir de uma descrição densa do episódio, pois parte-se do pressuposto de que as mensagens implícitas somente são detectadas a partir da desconstrução da narrativa. As categorias de análise são o figurino, as expressões gestuais, as cores, as trilhas sonoras e o conteúdo textual. Desta forma, através de Os Simpsons demonstra-secomo um desenho (visto de forma ingênua) pode conter mensagens implícitas de forma lúdica e bem humorada sobre a formação da opinião pública através das mídias \cite{pasquolotto}.

\subsection{Simpsons}

\chapter{Apocalíptico}

\chapter{Apocalípticos X Simpsons}

\chapter{Conclusão}
teste 

\postextual

\bibliographystyle{plain}

\bibliography{bibliografia}

\end{document}